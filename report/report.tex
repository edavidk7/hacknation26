% report.tex — Ruben: Hack-Nation 2026, VC Track
\documentclass[9pt,a4paper]{article}
\usepackage[utf8]{inputenc}
\usepackage[T1]{fontenc}
\usepackage[english]{babel}
\usepackage{geometry}
\geometry{left=8mm,right=8mm,top=8mm,bottom=8mm}
\usepackage{amsmath}
\usepackage{hyperref}
\usepackage{multicol}
\usepackage{enumitem}

\setlength{\parindent}{0pt}
\setlength{\parskip}{2pt}
\setlength{\columnsep}{5mm}

\title{\vspace{-12mm}\textbf{Ruben: Multimodal Music Composition via Tree-Structured Semantic Representation}\vspace{-6mm}}
\author{David Korcak, Frantisek Kmjec}
\date{}

\begin{document}
\maketitle

\begin{multicols}{2}
\footnotesize

\textbf{Abstract.}
Ruben transforms multimodal signals---text, images, video, audio---into music through a novel tree-structured semantic representation.
A multimodal LLM (Kimi K2.5) decomposes creative intent into an editable hierarchical tree, which is then assembled into a coherent prompt for ACE-Step 1.5 (text-to-audio diffusion).
Key contributions: (1)~a two-pass pipeline (analysis $\to$ human editing $\to$ assembly) that separates creative decomposition from prompt synthesis, (2)~an audio-to-text feedback loop enabling style transfer through natural language rather than learned embeddings, and (3)~an interactive tree editor providing interpretability and fine-grained control over generation.

\section{Architecture}

\textbf{Pipeline.} React frontend $\to$ FastAPI backend $\to$ ACE-Step 1.5 on remote GPU. Async job-based; end-to-end latency 45--90s (generation dominates at 30--60s).

\textbf{Inputs.} Text (direct prompt). Images (base64, Kimi vision extracts color palette, mood, spatial qualities $\to$ musical characteristics). Video (6 keyframes via OpenCV with temporal annotations). Audio references (ACE-Step \texttt{/lm/understand} extracts caption, BPM, key, lyrics $\to$ text representation for Kimi).

\textbf{Tree representation.} The core abstraction is a hierarchical JSON tree of musical characteristics with flexible, LLM-decided schema. Typical branches include Emotional Landscape, Instrumentation, Sonic Production, Temporal Dynamics, Harmonic Language, and Narrative Arc. Internally represented as recursive \texttt{SongNode} objects (name, value, metadata, children).

\subsection{Two-Pass Generation}

The critical insight: mechanical flattening of a tree into comma-separated tags produces incoherent prompts. Ruben instead uses two distinct LLM passes:

\textit{Pass 1 --- Analysis.} All inputs $\to$ Kimi + \texttt{SYSTEM\_PROMPT} $\to$ structured tree. The tree is rendered in the frontend for human inspection and editing.

\textit{Pass 2 --- Assembly.} Edited tree $\to$ Kimi + \texttt{ASSEMBLY\_PROMPT} $\to$ coherent caption + lyrics optimized for ACE-Step. The assembly pass resolves cross-branch convergences and tensions (e.g., when mood and genre imply conflicting tempos), specifies instruments with production-quality descriptors, and outputs a unified musical narrative.

\subsection{Audio-to-Text Feedback}

Reference audio $\to$ ACE-Step audio understanding $\to$ text caption + metadata $\to$ injected into Kimi's analysis context. This enables style transfer through the text-to-audio pathway: a lo-fi hip-hop reference produces trees with tape saturation, vinyl crackle, and relaxed swing---without requiring audio embeddings or fine-tuning.

\subsection{Models}

\textbf{Kimi K2.5} (via OpenRouter): long-context (128k+) multimodal LLM with text + vision, structured JSON output, and chain-of-thought reasoning. Serves two roles: tree analysis and prompt assembly.

\textbf{ACE-Step 1.5} \cite{ace}: open-source music diffusion model (DiT + LM). Turbo variant, 8-step inference, $<$2s per song on A100. Supports text-to-music, audio understanding, and metadata control (BPM, key, time signature, duration up to 600s).

\section{Interface}

Input panel: 2$\times$2 grid (text, audio, images, video) with drag-and-drop. Tree editor: tabbed \texttt{TreeStack} with hover toolbars, color-coded nodes, diff tracking against previous generations, and markdown export. Controls: duration (10--240s), BPM, key, time signature. History panel: previous generations with audio player, tree diffs, and restore.

\section{Results}

Multimodal inputs produce richer trees than any single modality. A frozen fjord photograph combined with ``lonely but hopeful'' text generated a D~Dorian ambient piece with felt piano, glacial pads, and a narrative arc from isolation to warmth---detail that neither input alone would produce. Reference audio measurably influences generation: a lo-fi hip-hop clip shifted instrumentation toward tape-saturated Rhodes and dusty drums. Tree editing propagates to audio: replacing ``felt piano'' with ``music box'' changed timbral character while preserving mood and structure. Assembly consistently outperforms mechanical flattening, which produces tag-soup prompts with no narrative coherence.

\section{Discussion}

\textbf{Contributions.} (1)~Tree-structured semantic representation as a reusable creative primitive. (2)~Two-pass generation separating decomposition from synthesis. (3)~Audio-to-text feedback enabling style transfer via natural language. (4)~Multimodal orchestration through a single LLM. (5)~Human-in-the-loop editing at arbitrary granularity.

\textbf{Limitations.} Duration capped at 240s for reliability (ACE-Step supports 600s). Async polling adds latency. Internet-dependent (Kimi via OpenRouter). Assembly can fail on contradictory inputs. Audio understanding limited for experimental music.

\textbf{Scale vision.} Trees become shareable templates, recommendation primitives (cluster by topology), and A/B testing units. Multi-gen stitching extends duration. Tree merging enables collaborative composition.

\section{Conclusion}

Ruben demonstrates that multimodal $\to$ semantic tree $\to$ audio is a viable paradigm for controllable AI music generation. The tree provides interpretability and editability absent from flat-prompt systems. Two feedback loops---audio-to-text style transfer and two-pass assembly---point toward a general architecture for human-AI creative collaboration.

\vspace{-2mm}
\begin{thebibliography}{8}
\setlength{\itemsep}{0pt}
\setlength{\parskip}{0pt}
\footnotesize
\bibitem{ace} Gong et al. ACE-Step 1.5. \url{https://github.com/ace-step/ACE-Step-1.5}, 2026.
\bibitem{kimi} Moonshot AI. Kimi K2.5. \url{https://kimi.moonshot.cn}, 2025.
\bibitem{openrouter} OpenRouter. \url{https://openrouter.ai}, 2025.
\bibitem{ddpm} Ho et al. Denoising Diffusion Probabilistic Models. NeurIPS, 2020.
\bibitem{gpt4} OpenAI. GPT-4 Technical Report. arXiv:2303.08774, 2023.
\bibitem{musiclm} Agostinelli et al. MusicLM. arXiv:2301.11325, 2023.
\bibitem{hifigan} Kong et al. HiFi-GAN. TASLP, 2020.
\bibitem{verma} Verma \& Chetty. Affective Music. ISMIR, 2018.
\end{thebibliography}

\end{multicols}
\end{document}
